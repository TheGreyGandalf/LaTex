\chapter[Simplesso]{}
\section{Ottimo in forma analitica}

\nt{
    Forma generale di un problema di LP:

    $Z(L P)=\operatorname{MAX} \sum_{s=1}^{n} c_j x_{j}, \sum_{j=1}^{m} x_{s} \leqslant b, i \epsilon\{1,2, \ldots, m\}, x_{j} \geqslant 0 s \in\{1,2, \ldots, m\}$

    Data una soluzione euristica, rispetta tutti i vincoli $\hat{x}$
    
    $\sum_{j=1}^{n} \operatorname{c}_{j}\hat{x}_j \leqslant Z(L P)$
}
\textbf{Linee di livello: punti in cui f è costante, i punti che possono essere ottimi in particolar modo per le linee sono i corner}
\ex{Esempio visto numerose volte in classe}{
    MAX $x_1+\frac{64}{100}x_2$ ,  n=2 variabili, m=2 vincoli, $f(x_1,x_2)=x_1+\frac{64}{100}x_2$

$$
50 x_{1}+31 x_2 \leqslant 250\\
-3 x_{1}+2 x_{2} \leqslant4\\
x_{1}, x_{2}>0
$$

Soluzione ottima: $\tilde{x}, z(L P)=\sum_{j=1}^{n} c_{j} \hat{x}_{j}$

$$
x=\left[\begin{array}{l}x_{1} \\ x_{2}\end{array}\right] \quad 
c=\left[\begin{array}{c}1 \\ \frac{64}{100}\end{array}\right] 
\quad A=\left[\begin{array}{cc}50 & 31 \\ -3 & 2\end{array}\right] 
\quad b=\left[\begin{array}{c}250 \\ 4\end{array}\right]
$$
%grafico

$N_z=(\tilde{x_1},\tilde{x_2})$

$\nabla f(x)=\frac{\delta f(x)}{f(x_1)}, \frac{\delta f(x)}{f(x_2)}=[1,\frac{64}{100}]$

$\tilde{x}=(\frac{396}{193},\frac{950}{193})$    $Z(LP)=\frac{984}{193}$
}
\section{Simplesso tramite metodo analitico}
Per arrivare alla forma standard di un problema LP è necessario aggiungere due 
variabili di “slack”, di modo da far comparire una uguaglianza, 
anziché un $\geq$, oltre a questo motivo si aggiungono n variabili perché per poter 
risolvere un sistema si devono avere m=n vincoli e variabili
\ex{Esempio}{
\begin{align*}
50 x_{1}+31 x_{2}+x_{3}+\qquad=250\\
-3 x_{1}+2 x_{2}\qquad+x_{4}=4\\
x_{1}, x_{2}, x_{3}, x_{4} \geqslant 0\\
\end{align*}

\begin{align*}
x_{3}=250-50 x_{1}-31 x_{2}\\ 
x_{4}=+4+3 x_{1}-2 x_{2}\\ 
-----------\\ 
z=0+x_{1}+\frac{64}{100}x_{2}
\end{align*}

Dizionario: Scritto in funzione della soluzione
$x_1=0,x_2=0,x_3=250,x_4=4$
$$
\begin{cases}x_{3}=250-250 x_{1} \longrightarrow 
    \underbrace{250+50 x_1}_{x_1\geq 0}\geqslant 0 \quad x_{1} \leq 5\\
    x_{4}=4+3 x_{1} \longrightarrow \underbrace{4+3 x_{1}}_{x_{4} \geqslant 0} \geqslant 0 
    \quad x_{1} \geqslant\frac{4}{3}    
\end{cases}
$$
Aumentando il valore di $x_1$ a 5, il valore di $x_3$ diventa 0
\begin{align*}
    x_{3}=250-50 x_{1}-31 x_{2} \longrightarrow x_{1}=5-\frac{31}{50} x_{2}-\frac{1}{50} x_{3}\\  
    x_{4}=4+3\left(5-\frac{31}{50} x_{2}-\frac{1}{50} x_{3}\right)-2 x_{2}\longrightarrow =19-\frac{199}{50} x_2-\frac{3}{50} x_3\\ 
    z=\left(5-\frac{31}{50} x_{2}-\frac{1}{50} x_{3}\right)+\frac{64}{100} x_{2}=5+\frac{1}{50} x_{2}-\frac{1}{50} x_{3}
\end{align*}
Nuovo dizionario:
\begin{align*}
    x_{1}=5-\frac{31}{50} x_{2}-\frac{1}{50 x_{3}}\\ 
    x_{2}-19-\frac{193}{50} x_{2}-\frac{3}{50} x_{3} \\
       ------------\\   
    z=5+\frac{1}{50} x_{2}-\frac{1}{50} x_{3} \quad
\end{align*}
$x_1=5,x_2=0,x_3=0,x_4=19$ Fissiamo x2, x3

$$
\begin{cases}
    x_{1}=5-\frac{31}{50}x_{2} \longrightarrow \underbrace{5-\frac{31}{50}}_{x_1\geq 0}\geqslant 0 \quad x_{1} \leq \frac{250}{31} \quad \\ 
    x_{4}=19+\frac{193}{50} x_{2} \longrightarrow \underbrace{19+\frac{193}{50} }_{x_{4} \geqslant 0} \geqslant 0 \quad x_{2} \leq\frac{950}{193}    
\end{cases}
$$
Incrementando $x_2$ a $\frac{950}{193}$
\begin{align*}
x_{4}=19-\frac{193}{50} x_{2}-\frac{3}{50} x_{3}\\
x_{2}=\frac{950}{193}-\frac{3}{193} x_{3}-\frac{50}{193} x_{4}
\end{align*}
Sostituisco:
\begin{align*}
    x_{1}=5-\frac{31}{50}\left(\frac{950}{193}-\frac{3}{193} x_{3}-\frac{80}{193} x_{1}\right)-\frac{1}{50} x_{3}=\frac{376}{193}-\frac{2}{193} x_{3}+\frac{31}{193} x^{4}\\ 
    Z=5+\frac{1}{50}\left(\frac{950}{193}-\frac{3}{193} x_{3}-\frac{50}{193} x_{4}\right)=\frac{964}{193}-\frac{38}{4825} x_{3}-\frac{1}{193} x_{4}
\end{align*}
\begin{align*}
    x_{1}=\frac{376}{193}-\frac{2}{193} x_{3}+\frac{31}{193} x_{4}, \quad \\ 
    x_2=\frac{950}{193}-\frac{3}{193}x_3-\frac{50}{193}x_4\\ 
    --------------\\
     z=\frac{984}{193}-\frac{98}{4825}-x_{3}-\frac{1}{193} x_{4}
\end{align*}
Soluzione del problema:
$x_{1}=\frac{376}{193}, x_{2}=\frac{950}{193},x_3,x_4=0$
%Fine del lungo esempio
}
\subsection{Revised Simpelex Algorithm}
\textbf{Forma generica di un problema LP in forma non standard}
$$
\begin{gathered}
    Z(L P)=\operatorname{MAX} \bar{C}^{\top} \bar{x} \\
    \bar{A} x \leqslant b \text{ (Abbiamo tanti vincoli quante righe)}\\
    x \geqslant 0
\end{gathered}
$$

Si portano in forma standard le n variabili $\quad x=\left\{x_{1}, x_{2}, \ldots, x_{n}\right\}$
Si aggiungono le variabili slack, cioè si aggiunge un vettore di m variabili
\begin{align*}
Z(LP)=C^Tx\\
A x=b \quad \text{m Vincoli}\\
x \geqslant 0
\end{align*}

\qs{Ipotesi}{
\begin{enumerate}
    \item $n>m$
    \item $rank(A)=m$
    \begin{enumerate}
        \item $m$ vincoli linearmente indipendenti
        \item $m$ variabili linearmente indipendenti
    \end{enumerate}
\end{enumerate}
}
\textbf{Idea principale}
\begin{enumerate}
    \item Assegnare il vettore 0 a n-m variabili (Non basic variables), così otteniamo un sistema di m equazioni in m variabili
    \item Esprimere il valore delle m variabili e della funzione obiettivo in funzione delle n-m variabili a 0
    \item Controllare se aumentare il valore delle n-m variabili
\end{enumerate}
Variabili settate a 0 $\rightarrow$ Prova di ottimalità (Ottimo locale), basata sul fatto che un LP è convesso
\textbf{Partizione delle variabili}
\begin{enumerate}
    \item Basic variables
    \begin{enumerate}
        \item $x_B \in B \subset x$ $\left\{\begin{array}{l}B \cap \bar{B}=0 \\B \cup \bar{B}=x\end{array}\right.$$m \text{ Variabili}, |B|=m$
    \end{enumerate}
    \item Non basic Variables
    \begin{enumerate}
        \item $x_{\bar{B}} \in \bar{B} \subset x$  $n-m \text{ Variabili}, |B|=n-m$
    \end{enumerate}
\end{enumerate}
$x=x_B, x_{\bar{B}}$

$$
\begin{aligned}
    & \colorbox{red}{$A_x=b$} \quad A_B x_B+A_{\bar{B}} x_{\bar{B}}=b \\
    & A_B x_B=b-A_{\bar{B}} x_{\bar{B}} \\
    & \fcolorbox{red}{white}{$x_B=A_B^{-1} b-A_B^{-1} A_{\bar{B}}-x_{\bar{B}}$}
\end{aligned}
$$
Dal secondo al terzo passaggio $A_B\cdot A_B^{-1}=I$ si moltiplica per l’inversa
Nel terzo passaggio è necessario calcolare la trasposta, anche se è complesso da computare, in $x_B$ alla fine sono conosciute le variabili fuori base conosciuta l’inversa

$c=x_B, x_{\bar{B}}$
$$
\begin{aligned}
    & \fcolorbox{red}{red}{$c^{\top}_x$}=c_B^{\top} x_B+c_{\bar{B}}^{\top} x_{\bar{B}} \\
    & c_B^{\top}\left(A_B^{-1} \cdot b-A_B^{-1} \cdot A_B \cdot x_{\bar{B}} \right)+c_B^{\top} x_{\bar{B}}= \\
    & \fcolorbox{red}{white}{$c_x^{\top} =c_B^{\top} A_B^{-1} b+\left(c_{\bar{B}}-c_B^{\top} \cdot A_B^{-1} A_{\bar{B}}\right) x_{\bar{B}} $}
\end{aligned}
$$

Se è conosciuta $A_B^{-1}$ si è a conoscenza del valore delle variabili 
di base e della funzione obiettivo.
$$
\text{Dizionario} \left\{\begin{array} { l l } { x _ { B } = \tilde { b } + \tilde { A } x _ { \overline { B } } } \\ 
    c^{\top} x=\psi + \tilde{c}^{\top}_{\bar{B}}x_{\bar{B}} \end{array}\right. \quad 
    \text { DEVE } \left\{\begin{array}{ll}\tilde{b}=A_B^{-1} b & \psi=c_B^{\top} A_B^{-} b \\ 
    \tilde{A}=-A^{-1}_BA_{\bar{B}}x_{\bar{B}} & \fcolorbox{red}{white}{$\tilde{C}_{\bar{B}}^{\top}=C_{\bar{B}}^{\top}-
    \fcolorbox{red}{red}{$C_B^{\top} A_B^{-1} $}A_{\bar{B}}$}\end{array}\right. 
$$

La parte cerchiata sono i costi ridotti, cioè un optimal simplex dictionary
 $\tilde{C}^{T}_B \leq 0$
Dato che le variabili in B fuori base sono a 0
$$
\begin{cases*}
    x_B=\tilde{b} \rightarrow \tilde{b} \rightarrow \text{Valore corrente della funzione obiettivo}  \\
    c^T x=\psi \rightarrow \psi \rightarrow \text{Valore delle variabili di base} 
\end{cases*}
$$

\nt{
\textbf{Dizionario}\newline
n variabili, m variabili libere

$$
\left(\begin{array}{l}m \\m\end{array}\right)=\frac{n !}{m !(n-m) !} \quad \text { Non si possono enumerare tutte}
$$
}

\dfn{Soluzione di base ammissibile}{
    Una soluzione di base ammissibile 
    $\quad x_{B}=\tilde{b} \geqslant 0$ ha questa forma
}

\dfn{Cambiamento di base}{
    $x_{p} \text{ con } \tilde{c}_{p} \geqslant 0$ pivot column

    \begin{itemize}
        \item     Variabili di base: $x_{B}=\tilde{b} \tilde{A}x_{B}$ una volta determinata $x_p$
    \end{itemize}
$$
\begin{aligned}
    & x_{B}=\tilde{b}+\tilde{A}_{P} x_{P} \\
    & \text{Calcolare } y=C_{B}^{\top} A_{B}^{-1} \\
    & y A_{B}=C_{B}^{\top} \longrightarrow y=C_{B}^{\top} A_{B}^{-1}
\end{aligned}
$$

La pivot colum $x_p$ sarà la prima variabile fuori base tale che:
$yA_P < c_p \longrightarrow \tilde{c}_p<0$
}

\ex{}{
$$
Max\ x_{1}+\frac{64}{100} x_2\\
\begin{aligned}
    & 50 x_{1}+3 x_{2}+x_{3} \ \ \quad=250 \quad m=4 \\
    & -3 x_{1}+2 n_{2} \quad+x_{4}=4 \quad m=2 \\
    & x_{1}, x_{2}, x_{3}, x_{4} \geqslant 0 \quad \ \ \ m-m=4-2=2
\end{aligned}
$$

$$
\begin{aligned}
    x=\left\{x_1, x_2, k_3, x_4\right\} \quad A 
    & =\left[\begin{array}{cccc}50 
    & 3 & 1 & 0 \\-3 & 2 & 0 & 1
    \end{array}\right] \quad b=\left[\begin{array}{l}250 \\
    4\end{array}\right] c^{\top} & =\left[
    \begin{array}{llll}1 & \frac{64}{100} & 0 & 0
    \end{array}\right]\end{aligned}
$$

$$
B=\{x_3,x_4\}, \bar{B}=\{x_1,x_4\}\\
$$

$$
A_B=\left[\begin{array}{ll}1 & 0 \\
0 & 1\end{array}\right] \quad A_{\bar{B}}=\left[\begin{array}{cc}50 & 31 \\
-3 & 2\end{array}\right] \quad C_B^{\top}=[0,0] \quad C_{\bar{B}}^{\top}=\left[\begin{array}{ll}1
&-\frac{64}{100}\end{array}\right]
$$

$$
\begin{aligned}
        & \left.A_B^{-1}=\left[\begin{array}{ll}1 & 0 \\
        0 & 1\end{array}\right]\right] \quad \tilde{b}=\left[\begin{array}{ll}1 & 0 \\
        0 & 1\end{array}\right] \cdot\left[\begin{array}{c}250 \\
        4\end{array}\right]=\left[\begin{array}{c}250 \\
        4\end{array}\right] & \text{valore var di base} \\
        & x_B=\left[\begin{array}{l}x_B \\
        x_A\end{array}\right]=\left[\begin{array}{c}250 \\
        a\end{array}\right] \\
        & \psi=[0,0]\left[\begin{array}{ll}1 & 0 \\
        0 & 1\end{array}\right]\left[\begin{array}{c}250 \\
        4\end{array}\right]=0\ {\text {f obiettivo per queste variabili base}} 
        \end{aligned}
$$

\[
\begin{aligned}
    \begin{array}{l}
        x_{1}=5-\frac{31}{50} x_{2}-\frac{1}{50} x_{3} \\
        x_{4}=19-\frac{143}{50} x_{2}-\frac{3}{50} x_{3} \\
        -----------\\
        z=5+\frac{1}{50} x_{2}-\frac{1}{50} x_{3}
        \end{array} \quad 
        \tilde{A}=
        \left[\begin{array}{cc}
        \frac{31}{50} & \frac{1}{50} \\
        \frac{193}{50} & \frac{3}{50}
        \end{array}\right] 
        \quad \tilde{b}=\left[
        \begin{array}{l}51 \\
        19\end{array}\right] 
        \end{aligned}\\
        5=\psi \\ 
        \tilde{C}_{\bar{B}}^{\top}=[\frac{1}{50}-\frac{1}{50}]\\
        B=\{x_1,x_4\}, \bar{B}=\{x_2,x_3\}\\
\]


È di nostro interesse solo la colonna di $x_2$, le altre non sono interessanti al fine di trovare i costi ridotto

$y=A_B=C^{\top}_B$ Per calcolare $y=C^{\top}_BA_B^{-1}$
}

\[
\begin{cases*}
    y_1 \ \ \ =0\\ 
    \ y_2\ =0 \\
\end{cases*}
\]

\(
\begin{aligned}
    & A_B=\left[
    \begin{array}{ll}
        1 & 0 \\
        0 & 1
    \end{array}\right] \\
    & y=\left[y_1, y_2\right]
    \end{aligned}\\
     \tilde{c}_{\bar{B}}^T=\left[
        \begin{array}{ll}
1, \frac{64}{100}
\end{array}\right]-\left[\fcolorbox{red}{white}{$
\begin{array}{cc}
0 & 0
\end{array}$}\right]\left[
    \begin{array}{ll}
50 & 31 \\
-3 & 2
\end{array}\right]=\left[\begin{array}{ll}
\overset{c_1}{1} & \overset{c_2}{\frac{64}{100}} 
\end{array}\right ]-\left[\begin{array}{ll}
0 & 0
\end{array}\right]=\left[\begin{array}{ll}
1 & \frac{64}{100}
\end{array}\right] \longrightarrow p=x_1
\)


\ex{}{
Appena $[1,\frac{64}{100}]> 0[\underset{-3}{0}]$ Confronto con quello dopo e appena trovo un valore positivo ci si ferma
Appena la matrice cerchiata diventa più piccola del costo, la si sceglie.

 Si mette $y=C^{\top}_BA_B^{-1}$, non siamo intenzionati a calcolare la trasposta, la si porta a 
 $y=C^{\top}_BA_B^{-1}$, una volta calcolata questa possiamo confrontare l’altra parte dei membri, 
 ottenendo anhe la funzione obiettivo $\psi=[0,0][\underset{4}{2500}]=[0,0]$

La pivot column entra in base, Pivot row esce dalla base.

Data la pivot column $A_Bd=A_p$    $d=A_B^{-1}A_P=\tilde{A}_p$

$A_p$ è: $A p=\left[\begin{array}{c}
    50 \\-3\end{array}\right] d=\left[
        \begin{array}{l}
        d_1 \\
        d_2\\
        \end{array}\right] \quad
            \begin{cases*}
                d_1=50 \\
                d_2=-3
            \end{cases*}
            \begin{array}{l}

        \end{array}$
        Questo è il pezzo di $\tilde{A}$ interessa

$\tilde{b}=td>0$ Trovare il valore massimo di t facendo salire a t la variabile di pivot

$$
\begin{array}{llll}\tilde{b_1}-td_{1} \geqslant 0 & 250-t*50 \geqslant 0 & t \leqslant 5 \\\tilde{b}_2-td_{2} \geqslant 0 & 4-t(-3) \geqslant 0 & t \geqslant-\frac{4}{3}\end{array} \mid
$$

$\underset{\text{=5 nuovo }x_1}{t^*}$ è il nuovo valore di $x_p$ le altre variabili $\tilde{b}-td$        $x_4=19$

}

\textbf{Per far partire l’algoritmo del simplesso si fanno comparire una entità alla destra della matrice A}

L’algoritmo del simplesso è migliorabile se dall’iterazione precedente è possibile migliorare ciò che si ha.

In alcuni casi l’algoritmo può ciclare, per esser sicuri che ciò non accada è necessario prendere le variabili sempre in ordine
 (\textbf{Regola di Bland}), fino ad arrivare in fondo, se un ciclo si presenta è un campanello d’allarme che non sono state prese
le variabili in ordine corretto

