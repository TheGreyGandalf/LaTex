\chapter{}
\section{Problema duale}
Problema duale del problema già affrontato, in questo caso lo scopo è trovare degli Upper-BoundProblema duale del problema già affrontato, 
in questo caso lo scopo è trovare degli Upper-Bound

\ex{}{
$$
\text{Max }4 x_1+x_2+5 x_3+3 x_4\\\left(x_1\right) x_1-x_2-x_3+3 x_4 \leqslant 1 \\\left(y_2\right) 5 x_1+x_2+3 x_3+8 x_4 \leqslant 55 \\{ }{\left(y_3\right)}-x_1+2 n_2+3 n_3-5 n_4 \leq 3
\\ x_1, x_2, x_3, x_4 \geqslant 0 
$$

$
\hat{x_1}=1,\hat{x_2}=0,\hat{x_3}=0,\hat{x_4}=0,z=4 \text{ Lower-Bound}
$

y1,y2,y3 sono moltiplicatori dei vincoli, per ottenere un Bound valido, essi devono essere $\geq$ della funzione obiettivo

\textbf{Moltiplicando la seconda riga per $\large \frac{5}{3}$}

$\Large \frac{25}{3} x_1+\frac{5}{3} x_2+5 x_3+\frac{40}{3} x_4 \leq \frac{275}{3}$ I termini sono tutti $\leq$ della funzione Obiettivo

$\Large 4 x_{1}+x_{2}+5 x_{3}+3 x_{4} \leqslant \frac{25}{3} x_{1}+\frac{5}{3} x_{2}+5 x_{3}+\frac{40}{3} x_{4} \leqslant \frac{275}{3}$ 
Questo valore è un Upper-Bound, Non è possibile moltiplicare per un termine negativo, poiché la disuguaglianza si andrebbe a girare di segno.

\textbf{Andando a sommare la seconda e la terza riga}

$\Large 4 x_{1}+3 x_{2}+6 x_{3}+3 x_{4} \leqslant 58 \rightarrow$ Miglior Upper-Bound, 58 è il maggiore

Upper-bound migliori si ottengono andando a ricercare combinazioni lineari dei vincoli quando abbiamo una sima per eccesso della funzione obiettivo

Vincoli $\rightarrow$ si moltiplica o si somma $\rightarrow$ facendo si che i coefficienti siano sopra la sunzione obiettivo

\begin{align*}
    y_{1}+5 y_{2}-9_{3} \geqslant 4\\
    -y_{1}+y_{2}+2 y_{3} \geqslant 1\\
    -y_{1}+3 y_{2}+3 y_{3} \geqslant 5\\
    y_{1},y_{2},y_{3},y_{4}\geqslant 0
\end{align*}
4,1,5 Sono i coefficienti della funzione obiettivo
(Si ha ribaltato i vincoli di 90° per ottenere il duale)

$$
\left(y_{1}+5 y_{2}-y_{3}\right) x_{1}+\left(-y_{1}+y_{2}+2 y_{3}\right) x_{2}+\left(-y_{1}+3 y_{2}+3 y_{3}\right) \\
x_{3}+\left(3 y_{1}+8 y_{2}-y_{y_{3}}\right) x_{4} \leq y_{1}+55 y_{2}+3 y_{3} \leadsto \ \text{Coefficienti dei vincoli, Da minimizzare per il miglior Upper-Bound}
$$
}

Questi passaggi restituiscono un Upper-Bound, è necessario minimizzare questo tra i moltiplicatori 
per cercare un \textbf{vero Upper-Bound.}

\nt{
    Forma generale di un problema duale LP
    \begin{align*}
        & Z(D(L P))=MIN \sum_{i=1}^{m} b_{i} y_{i} \\
        & { }_{A}^{\operatorname{Vincolato}} \sum_{i =1}^{m} a_{i j} y_{i} \geqslant c_{j}, \forall j \in\{1,2, \ldots, n\} \\
        & y_{i} \geq 0 \quad \forall i \in\{1,2, \ldots, m\}
    \end{align*}
}

\clm{Regola generale passaggio Primale-Duale}{}{
    \begin{itemize}
        \item Variabili libere $\rightarrow$ Vincolo di uguaglianza
        \item Variabile $\geq$ 0 $\rightarrow$ Vincolo $\geq$ 0
        \item Vincolo $\leq$ 0 $\rightarrow$ Variabile $\geq$0
        \item Vicolo = $\rightarrow$ Variabile libera
    \end{itemize}
}

\ex{}{
    \begin{align*}
        Max \ 2 x_{1}-4 x_{2}\\
        \left(y_{1}\right)\ -x_{1}+x_{2} \leqslant 2 \\
        (y_2) 8 x_{1}+2 x_{2} \leq 19
    \end{align*}

n=2 m=2: Il punto di intersezione ($\tilde{x}$) viene calcolato utilizzanzo la formula
del punto di incostro date due equazioni in n variabili 


$\large \tilde{x}=\left[\frac{3}{2}, \frac{7}{2}\right]$
%immagine

Dati:
\[
x=\left[\begin{array}{l}
    x_{1} \\
    x_{2}\end{array}
    \right] \quad c=\left[
        \begin{array}{l}
            2 \\
            4\end{array}\right] \quad b=\left[
            \begin{array}{c}2 \\
            19 \end{array}\right] \quad A=\left[
            \begin{array}{cc}
                -1 & 1 \\
                8 & 2
            \end{array}\right]
\]
Variabili primali:
\[
\text { MIN } 2 y_{1}+19 y_{2} \\-y_{1}+8 y_{2} \geqslant 2 \\y_{1}+2 y_{2} \geqslant 4 \\y_{1},y_{2} \geqslant 0 
\]

Duale $\rightarrow$ Upper senza simplesso

$\large \tilde{y}=\left[\frac{14}{5}, \frac{3}{5}\right]$
%Immagine
}

\nt{
%Da sistemare, non bello matrici così sparse
    \textbf{Forma matriciale}
\[
y=\begin{bmatrix}
    y_1 \\
    y_2\\ 
    \vdots\\
     y_n
    \end{bmatrix}   
    \]
    \[
    \begin{matrix}
        \quad A^Ty \geqslant c \\
     \quad y A \geqslant c
    \end{matrix}
    \]
    Vincoli del duale\quad 
    \[
    c=
    \begin{bmatrix}
            c_1 \\
            c_2 \\
            \vdots \\
            c_n
    \end{bmatrix}
    \]
    \[
        b=
        \begin{bmatrix}
                b_1\\
                b_2 \\
                \vdots \\
                b_n
        \end{bmatrix}
        \]
\quad $b^Ty$ \ \ Funzione Obiettivo

}

\nt{
    \textbf{Forma generale vincoli in forma non duale}
        \begin{align*}
            z(L P)=\min \sum_{i=1}^n \sum_{j=1}^m c_{i j} x_{j}\\
            \sum_{j=1}^{n} x_{i j} \leq a_{i}, \forall i \in \{1,2, \ldots,m\}(\lambda_{i})\\
            \sum_{i=1}^{m} x_{i j} \geqslant b_j, \forall j \in \{1,2,..., m\} (\pi_j)\\
            \forall\ x_{i j} \geq 0\ \forall \ i \ j\\
        \end{align*}
}
m e n non sono più il numero di vincoli e di variabili, ma ora indicano stazioni e depositi, nell’esempio

\subsection{Da problema di massimo a problema di minimo}
\ex{}{
\textbf{Massimo}

\[
\begin{matrix}
    300x_{11}&+200x_{12} & +100x_{13}&+250x_{21}&+400 x_{22}&+80 x_{23} \\ 
     x_{11}&+x_{12} &+x_{13} &&&& \leqslant 1200,5\ (\lambda_1) \\ 
     &&&x_{21}&+x_{22}&+x_{23} &\leqslant 1100,5(\lambda_2) \\ 
     x_{11}&&&+x_{21}&&& \leqslant -500,5 \ (\pi_1) \\
      & x_{12}&&&+x_{22}&& \leqslant -600,5\ (\pi_2) \\ 
      &&x_{13}&&&+x_{23}& \leqslant -1000,5 (\pi_3)\\
      x_{11}, x_{12}, x_{13}, x_{21}, x_{22}, x_{23} \geqslant 0\\
\end{matrix} 
\]\newline 

Per ottenere il minimo è necessario “ribaltare” la matrice utilizzando come variabili $\pi$ e $\lambda$

\textbf{Minimo}

\[
\begin{matrix}
1200,5 \lambda_1+1100,5 \lambda_2-500,5 \pi_1-600,5 \pi_2-1000 \pi_3\\  
    \lambda_1 &&-\pi_1&&& \geqslant-300 \\
     \lambda_1&&&-\pi_2 && \geqslant-200 \\
      \lambda_1&&&&-\pi_3 &\geqslant-100 \\
       &\lambda_2&-\pi_1 &&&\geqslant-250 \\
        &\lambda_2 &&-\pi_2 &&\geqslant-400 \\
         &\lambda_2 &&&-\pi_3 & \geqslant-80 \\
        \lambda_1, \lambda_2, \pi_1, \pi_2, \pi_3 \geqslant 0\\
\end{matrix}
\]
\newline

$\lambda_1=0, \lambda_2=0, \pi_1=250, \pi_2=200, \pi_3=80 \\
-500,5 \cdot 250-600,5 \cdot 200-1000,5 \cdot 80=-325268$

Il numero trovato è un upper-bound del problema, cioè: 325265 $leq$ Z(LP)

\[
\begin{array}{c}
-Z (O (L P))=min \sum_{i=1}^n a_i \lambda_i+\sum_{j=1}^m-b_j \pi_j \\
\lambda_i-\pi_j \geqslant-c_{i j} \quad \forall i \epsilon\{1,2, \ldots, n\},\quad \forall j \epsilon\{1,2, \ldots, m\} \\
\lambda i \geqslant 0 \quad \forall i \in\{1, 2, \ldots, m\} \\

\pi_j \geqslant 0 \quad \forall j \in\{1,2, \ldots, m \mid \}\\
\end{array}
\]
}

\mlenma{Il duale del duale coincide col primale}{

\[
MAX \sum_{j=1}^n c_j x_j \longrightarrow\\\sum_{j=1}^{m} -a_{i j} x_j \leqslant b \quad i \in\{1, 2, \ldots, m\} \\ x_j\geq 0, \quad i \in\{1, 2, \ldots, m\}   \\ 
\]

\[
MIN \sum_{i=1}^m b_{j} y_i \longrightarrow\\ \sum_{i=1}^m a_{i j} y_i \geqslant c_j, \forall \in\{1,2, \ldots, n \}\\y_i>0, j \in\{1,2, \ldots, m\} 
\]

\[
MAX \sum_{i=1}^m-b_i y_i \longrightarrow\\ min \sum_{j=1}^n -a_{i j} x_j \leqslant-c_j \quad \forall i \in\{1, 2, \ldots, n\}\\ \\y_i>0, j \in\{1,2, \ldots, m\} 
\]
\[
MAX \sum_{j=1}^n -c_j x_j \longrightarrow\\\sum_{j=1}^{m} -a_{i j} x_j \geqslant -b \quad \forall i \in\{1, 2, \ldots, m\} \\ x_j\geq 0, \quad j \in\{1, 2, \ldots, m\}\\
\]
\[
\longrightarrow MAX \sum_{j=1}^n c_j x_j \\\sum_{j=1}^{m} -a_{i j} x_j \leqslant b \quad i \in\{1, 2, \ldots, m\} \\ x_j\geq 0, \quad i \in\{1, 2, \ldots, m\}   \\
\]
}

\nt{
    \textbf{Forma generica di un LP e del duale }

    \begin{itemize}
        \item n variabili
        \item \begin{itemize}
            \item m1 variabili non negative
            \item m-m1 variabili libere
        \end{itemize}
        \item m vincoli
        \item \begin{itemize}
            \item m1 disuguaglianze
            \item m-m1 disuguaglianze
        \end{itemize}
    \end{itemize}

\[
    \begin{array}{c}
z(LP)= MAX \sum_{j=1}^{n} c_{j} x_{j}\\ 
(y_i)\ \sum_{j=1}^{n} a_{i j} x_j \leqslant b_i \quad i \in\{1, 2, \ldots, m_1\} \\
(y_i)\ \sum_{j=1}^{n} a_{i j} x_j = b_i \quad i \in\{m_1+1, 2, \ldots, m\} \\
  x_j\geq 0, \quad j \in\{1, 2, \ldots, m\} \\
    \end{array}
\]

$$
\Updownarrow
$$

\[
    \begin{array}{c}
z(D(LP))= MIN \sum_{j=1}^{m} b_{i} y_{i}\\ 
(x_j)\ \sum_{j=1}^{m} a_{i j} y_j \geqslant c_j \quad \forall j \in\{1, 2, \ldots, n_1\} \\ 
(x_j)\ \sum_{j=1}^{m} a_{i j} y_j = c_j \quad \forall j \in\{n_1+1, 2, \ldots, n\} \\
 y_i\geq 0, \quad \forall i \in\{1, 2, \ldots, m_1\}   \\
    \end{array}
\]
}

\subsubsection{Proprietà del duale}
Non si massimizzano le costanti, per esempio $Max \sum \alpha_i$ non ha senso

\textbf{Generazione di ulteriori vincoli da quelli già disponibili:}

Se si volessero generare altri vincoli validi si potrebbe procedere andando a sommare i vincoli già esistenti 
anche moltiplicati per una costante.

\ex{}{
\[
    \begin{array}{c}
        (y_1) 2 \cdot\left(2 w_{1}+3 x_{2}\right) \leqslant 3 \cdot 2\\
\left(y_{1}\right) 2 \cdot\left(7 x_{1}+4 x_{2}\right) \leqslant 7 \cdot 2\\ \downarrow \\ 18 x_{1}+14 x_{2} \leqslant 20
    \end{array}
\]
}

\thm{Teorema della dualità debole}{
    Dati $\left(\bar{x}_{1}, \bar{x}_{2}, \ldots, \bar{x}_{n}\right)$ soluzione ammissibile primale 
    e $*\left(\bar{y}_{1}, \bar{y}_{2}, \ldots, \bar{y}_{m}\right)*$ soluzione ammissibile duale, abbiamo: $\sum_{s=1}^n c_j \bar{x}_j \leqslant \sum_{i=1}^m b_i \bar{y}_i$
}

\begin{myproof}[Dimostrazione dualità debole]
    Per ogni coppia di soluzioni ammissibili primale-duale abbiamo:

    $$
    \sum_{j=1}^{n} c_{j} \bar{x}_{j} \leqslant \sum_{I=1}^{n}\left(\underset{\text{Ammissibilità duale}}{\sum_{i=1}^{m} a_{i j} \bar{y}_{i}}\right) \bar{x}_{j}=\sum_{i=1}^{m}\left(\underset{\text{Ammissibilità primale}}{\sum_{j=1}^{n} a_{i j} \bar{x}_{j}}\right) \bar{y}_{i} \leqslant \sum_{i=1}^{m} b_{i} \bar{y}_{i}
    $$

    \begin{itemize}
        \item Ammissibilità duale
        \begin{itemize}
            \item $c_j \leqslant \sum_{j=1}^{n} a_{i j} \bar{x}_{J}$
        \end{itemize}
        \item Ammissibilità primale
        \begin{itemize}
            \item $\sum_{j=1}^{n} a_{i j} \bar{x}_{j} \leqslant b_i$
        \end{itemize}
    \end{itemize}
\end{myproof}

\thm{Teorema della dualità forte}{
Dati $\left(\tilde{x}_{1}, \tilde{x}_{2}, \ldots, \tilde{x}_{m}\right)$ soluzione ottima primale, $\left(\tilde{y}_{1}, \tilde{y}_{2}, \ldots, \tilde{y}_{m}\right)$ soluzione ottima dove abbiamo: 

$$
\quad \sum_{j=1}^{n} c_{j} \tilde{x}_{j}=\sum_{i=1}^{m} b_{i} \tilde{y}_{i} \ \ (A)
$$
}

\begin{myproof}
    Costruiamo un duale $\left(\tilde{y}_{1}, \tilde{y}_{2}, \ldots, \tilde{y}_{m}\right)$ che soddisfi (A), allora per la dualità debole è ottima.

    Dal primale in forma canonica introduco le variabili di slack $\fcolorbox{red}{white}{$(*)$}$
    
    $x_{m+i}=b_{i}-\sum_{j=1}^{m} a_{ij}  x_{j}, \forall i \in\{1,2, \ldots, m\}$
    
    Eseguendo l’algoritmo del simplesso: $z(LP)=z^*+\sum^{n+m}_{k=1}\tilde{c}_kx_k$ con $z^*$ si ottiene l’ultima riga del $\fcolorbox{red}{white}{$(B)$}$ dizionario ottimo
    
    Dove $\tilde{C}{k} \leqslant 0, \forall k \in\{1,2, \ldots, m+m\} ( \tilde{C}{k}=0$ se variabile di base)    

    Definiamo $\fcolorbox{red}{white}{$(**)$}$

$\fcolorbox{red}{white}{$\tilde{y}_{1}=-\tilde{C}_{n+i}$}, \forall\ i \in\{, \ldots, m\}$

Quindi abbiamo da $\fcolorbox{red}{white}{$B$}$, usando $\fcolorbox{red}{white}{$*$}$ ed $\fcolorbox{red}{white}{$**$}$ :

\[
    \begin{array}{c}
\sum_{J=1}^m c_j x_j=z^*+\sum_{j=1}^n \tilde{c}_j x_j-\sum_{i=1}^m \underset{\tilde{y}_i=-\tilde{cj_{n+1}}}{\tilde{y}_{i}}\underset{x_{n+1}}{\left(b_i-\sum_{j=1}^n a_{i j} x_j\right)} \\
\sum_{j=1}^n c_j \fcolorbox{red}{white}{$x_j$} =\left(z^*-\sum_{i=1}^m b_i \tilde{y}_i\right)+\sum_{j=1}^n\left(\tilde{c}_j+\sum_{i=1}^m a_{i j} \tilde{y}\right) \fcolorbox{red}{white}{$x_j$}\\
 \text{Valida per ogni scelta} \\
 \downarrow \\
  \text{Quindi abbiamo che:}
    \end{array}
\]

(Perché si verifichi l’uguaglianza si devono verificare 2 condizioni, che la prima parte sia uguale a 0, la seconda sommatoria sia uguale a 0)

\[
    \begin{array}{c}
        (I)\ z^{*}=\sum_{i=1}^{m} b_{i} \tilde{y}_{i} \quad\\
         \text{Quindi dalla dualità debole}\\
          \text{Abbiamo una soluzione ottima}
    \end{array}
    \begin{array}{c}
(II)\ c_{J}=\tilde{c_{J}}+\sum_{i=1}^{m} a_{i j} \tilde{y}_{i}, \forall\ j\ \in \{1, \ldots, m\}\\ 
\text{Dato che }\tilde{c}_k \leqslant 0\\
 \sum_{i=1}^m a_{ij} \tilde{y}_i \geqslant c_j \in \tilde{y}_i \geqslant 0, \forall i \in \{1,\ldots, m \}\\
  \text{Quindi } \tilde{y} \text{ è una soluzione ammissibile}
    \end{array}
\]

\[
    \begin{array}{c}
    \text{Quindi}\\
     \Downarrow\\
    z^{*}=\sum_{j=1}^{n} C_{j} \tilde{k}_{j}=\sum_{i=1}^{m} b_{i} \tilde{y}_{i}\\
    \end{array}
\]

\end{myproof}

\cor{Scarti complementari: (corollario di dualità forte e debole)}{
Può essere espresso in 3 versioni, equivalenti tra loro:

Siano $\left(\tilde{x}_{1}, \tilde{x}_{2}, \ldots, \tilde{x}_{n}\right)$ una soluzione ammissibile primale
 e $\left(\tilde{y}_{1}, \tilde{y}_{2}, \ldots, \tilde{y}_{m}\right)$ una soluzione ammissibile duale, allora 
 condizioni necessarie e sufficienti (se e solo se) e sufficienti all’ottimalità di entrambe sono:
 \begin{enumerate}
    \item Prima versione (A):
    \begin{enumerate}
        \item $\sum_{i=1}^m a_{i j} \cdot \tilde{y}i=c_j \text { oppure}\ \tilde{x}_j=0 \text { (o entrambi) } \forall\ j \in\{1, \ldots, n\}$
        \item $\sum_{i=1}^n a_{i j} \tilde{x}_j=b_i\ \text{oppure} \ \tilde{y}_j=0 \ \text{(o entrambi)} \quad \forall i \in\{1, \ldots, m\}$
    \end{enumerate}
    \item Seconda versione (B):
    \item \begin{enumerate}
        \item $\tilde{x}_j\left(c_j-\sum_{i=1}^m a_{i j} \tilde{y}_i\right)=0$
        \item $\tilde{y}_i\left(\sum_{i=1}^m a_{i j} \tilde{x}_j-b_i\right)=0$
    \end{enumerate}
    \item Terza versione (C):   Data $\left(\tilde{x}_{1}, \tilde{x}_{2}, \ldots, \tilde{x}_{n}\right)$ 
    soluzione ammissibile primale, è soluzione ottima se e solo se esiste un vettore $\left(\tilde{y}_{1}, \tilde{y}_{2}, \ldots, \tilde{y}_{m}\right)$ tale:

    \begin{enumerate}
        \item $\sum_{i=1}^m a_{i j} \tilde{y}_i=c_j \operatorname{se} x_j>0 \quad \forall j \in\{1, \ldots,n\}$
        \item $\tilde{y}=0\ \text{se} \sum_{j=1}^n a_{i j} \tilde{x}_i<b_i \quad \forall i \in\{1, \ldots,m\}$
        \color{red}
        \item $\sum_{i=1}^m a_{i j} \tilde{y}_i \geqslant c_j \quad \forall j \in\{1, \ldots, n\}$
        \item $\tilde{y}_i \geqslant 0 \quad \forall i \in\{1, \ldots, m\}$
        ammissibilità duale
    \end{enumerate}
 \end{enumerate}
}
