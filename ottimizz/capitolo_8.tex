\chapter[Knapsack problem]{}
\section{}
\textbf{Istanza del problema}

\dfn{}{

\begin{enumerate}
    \item Insieme N=\{1,2,\dots,n\} di n oggetti
    \item Profitto $P_j$ ($P_j>0 \in \mathbb{Z}_+$)ed un peso $X_j \geq 0 \in \mathbb{Z}_+$
    \item Capacità dello zaino $c \in \mathbb{Z}_+$ 
 \end{enumerate}

 \textbf{Obiettivo:}

 Il Knapsack problem richiede di trovare un sottoinsieme di oggetti $S^* \subseteq N$ di profitto massimo e peso minore o uguale alla capacità dello zaino.
 Dato un sottoinsieme $S \subseteq N$ di oggetti:

 \[
    P(S)=\sum_{J \in S} P_J \quad W(J)=\sum_{J \in S} W_J
\]

Trovare $S^* \subseteq N$ tale che $P(S^*)$ è massimo e $W(S^*) \leq c$
}

Il costo di queste operazioni si attesta di $2^n$, dove n è il numero di oggetti da massimizzare

\[
    x_j=\left\{
        \begin{array}{ll}
        1 & \text { if item } j \text { is selected } \\
         0 & \text { otherwise }
        \end{array} \forall j \in N=\{1,2, \ldots, n\}
         \right.
\]

\clm{$\operatorname{ILP}_{\mathrm{KP}}$}{}{
    \begin{equation}
        \begin{aligned}
        & z\left(\mathrm{ILP}_{\mathrm{KP}}\right)=\max \sum_{j \in N} p_j x_j \quad \text{Quando è selezionato}\\
        & \text { subject to } \sum_{j \in N} w_j x_j \leq c \text {, } \quad \text{Peso $\leq$ Capacità}\\
        & x_j \in\{0,1\}, \quad \forall j \in N . \\
        &
        \end{aligned}
        \end{equation}
}

\ex{}{

\begin{equation*}
    \begin{array}{l|cccl}
    \hline \text { item } & p_j & w_j & \frac{p_j}{w_j} & \text { efficiency } \\
    \hline j=1 & 6 & 2 & \frac{6}{2} & =3 \\
    j=2 & 5 & 3 & \frac{5}{3} & \approx 1.66 \\
    j=3 & 8 & 6 & \frac{8}{6} & \approx 1.33 \\
    j=4 & 9 & 7 & \frac{9}{7} & \approx 1.28 \\
    j=5 & 6 & 5 & \frac{6}{5} & =1.2 \\
    j=6 & 7 & 9 & \frac{7}{9} & \approx 0.77 \\
    j=7 & 3 & 4 & \frac{3}{4} & =0.75 \\
    \hline
    \end{array}
\end{equation*}

\begin{equation*}
    \begin{array}{rrrrrrrrr}
    \max \quad 6 x_1&+5 x_2&+8 x_3&+9 x_4&+6 x_5&+7 x_6&+3 x_7 & \\
    2 x_1&+3 x_2&+6 x_3&+7 x_4&+5 x_5&+9 x_6&+4 x_7 \leq & 9 \\
    x_1, & x_2, & x_3, & x_4, & x_5, & x_6, & x_7 \in & \{0,1\}
    \end{array}
\end{equation*}
Una soluzione ottima è:

$$
x_1^*=1, x_2^*=0, x_3^*=0, x_4^*=1, x_5^*=0, x_6^*=0, x_7^*=0
$$
Il valore della soluzion ottima è $z( ILP_{KP} ) = 15$

\textbf{Soluzione Euristica greedy}

\[
    \begin{array}{c}
        x_1=1,x_2=2, x_7=1\\
        W(\{1,2,7\})=9 \quad P(\{1,2,7\})=14 \text{ Non ottima!}\\
        \text{La soluzione ottima sarebbe $x_1=1,x_4=1, z( ILP_{KP} ) = 15$}
    \end{array}

\]

}

