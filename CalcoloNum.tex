\documentclass{article} % \documentclass{} is the first command in any LaTeX code.  It is used to define what kind of document you are creating such as an article or a book, and begins the document preamble

\usepackage{amsmath} % \usepackage is a command that allows you to add functionality to your LaTeX code
\usepackage{pythonhighlight}
\usepackage[]{tikz}
\usepackage{fancybox}
\usepackage{amssymb}
\usepackage{pgfplots}
\usepackage{xskak}   
\usepackage{matlab-prettifier}

% The preamble ends with the command \begin{document}
\begin{document} % All begin commands must be paired with an end command somewhere
\begin{center}
    
    \newpage

    \section*{Metodo dei residui}

        \(Ax=b, \ A_{nxn}\)\\
        \(Ax^{(k)}-b =r^{(k)}\)\\
        \(\downarrow\)\\
        \(Ax^{\underset{\text{Soluzione esatta}}{*}}-b=0\)\\

    Il residuo è il vettore nullo \(\rightarrow \ x=x^* \) \\
    In ogni altro caso \(A\tilde{x} \neq b \rightarrow A\tilde{x} - b \neq 0\)\\
    Cioè \(\tilde{x} \neq x^*\)\\

    \textbf{Proprietà} Sia fissato \(\tau >0\)\\
    Se \(\frac{||r^{(k)}||}{||b||}\leq \tau\) allora \(\frac{||x^{(k)-x^*}||}{||x^*||}\leq K(A)\tau\)
    \(\frac{||x^{(k)-x^*}||}{||x^*||}\leq K(A) \underset{\text{Supposto}\le \tau}{\frac{||r^{(k)}||}{||b||}}\leq \tau \Rightarrow \frac{||x^{(k)-x^*}||}{||x^*||}\leq K(A)\tau\)
    \\
    \(||x^{(x)}-x^*|| \leq \epsilon ||x^{(k+1)}||\)

    \begin{lstlisting}%[mathescape=true]%

        for K=0,1,2, ... , Kmax
            r=Ax_0-b
            x=Gx_0+c
            if  \frac{||x-x_0||}{||x||} \lt \tau \and \frac{||r||}{||b||} <lt \tau 
                return x_0
            end
            x_0=0
        end
    \end{lstlisting}

        
    
    
    M è simile ad A, \(M^{-1} ~ G ~ O\)\\
    \(G=(I-M^{-1}A)\)\\
    Allora M ha una "struttura semplice": diagonale triangolare superiore oppure triangolare inferiore\\
    \(Mx=Mx-Ax+b\)\\
    \
    \(Mx^{k+1}=\underbrace{(M-A)x^{(k)}+b}_{\Sigma^{k} \in \mathbb{R}^n}\)\\
    \(Mx^{k+1}=\Sigma^{(k)}\)

    \subsection*{Metodi di decomposizione}

    A=\(\begin{pmatrix}
        a_{11} & a_{12} & a_{13} & ... & a_{1n}\\
        a_{21} & a_{22} & a_{23} & ... & a_{2n}\\
        a_{31} & a_{32} & a_{33} & ... & a_{3n}\\
        ...\\
        a_{n1} & a_{n2} & a_{n3} & ... & a_{nn}
    \end{pmatrix}\)



\end{center}
\end{document} % This is the end of the document